\documentclass[onecolumn]{aastex61}   	% use "amsart" instead of "article" for AMSLaTeX format
\usepackage{geometry}                		% See geometry.pdf to learn the layout options. There are lots.
\geometry{letterpaper}                   		% ... or a4paper or a5paper or ... 
\usepackage{graphicx}				% Use pdf, png, jpg, or eps§ with pdflatex; use eps in DVI mode
\usepackage{amsmath}
\usepackage{amssymb}
\usepackage{natbib}
\usepackage{lineno}
\usepackage{color}
\linenumbers

\begin{document}

\title{DESI Observations of Type Ia Supernovae Discovered by Imaging Surveys}

\begin{abstract}
The DESI survey will be operational concurrently with imaging surveys such as ZTF and LSST.  These imaging surveys are more
efficient than DESI in discovering transients due to both higher survey solid-angle coverage and cadence.  Imaging surveys, however,
do not provide accurate classification nor redshifts: features that a spectroscopic instrument such as DESI can provide. The DESI Time Domain
Working Group is therefore determining the potential science yield of a joint DESI--imaging Type~Ia supernova (SN~Ia) program.
\end{abstract}

\section{Fiducial Imaging Survey Plus DESI Program}
Let us consider what happens to one supernova. The supernova is characterized by its intrinsic properties (for example, light-curve
shape, color), its redshift, and is associated with a host galaxy.  It explodes on some observer-frame date $t_0$, and has
an average incident flux of $m_X(t)$ through filter $X$ that gets to the observer. 

An imaging survey takes $N$ observations covering the position of the supernova, at times $t_i$ and filters $X_i$, where $i$ is the observation index.
Transients are searched for in these images, so that the supernova has an opportunity of being detected.
To first order, the probability of detecting a supernova
in observation $i$ is a function of the supernova magnitude  $\epsilon_i(m_{X_i}(t_i))$.  The probability of supernova
discovery using a single detection is
\begin{equation}
\epsilon^D = 1- \sum_{i=1}^{N} \left(1-\epsilon_i(m_{X_i}(t_i))\right),
\end{equation}
where the sum is over all $N$ observations.  If  multiple detections are required for discovery, the efficiency is the sum of all
combinations of detections that allow for discovery.
If the supernova is indeed discovered, the date of discovery $t_D$ is stochastic.  
 
The imaging survey produces light curves of the supernova, from which (together with redshift) its distance can be determined with precision $\sigma_\mu$.
A poor distance determination occurs if the imaging survey does not observe the rise or fall of the supernova.  Therefore,
the supernova is of interest only if its date of explosion $t_0$ occurs from shortly before sky monitoring, to $\sim 2$ supernova-frame months
before the end of the monitoring. 
 
If the supernova is discovered, there is an opportunity for its targeted DESI observation, and/or the targeted
observation of its host galaxy when the transient light has faded away.  
The decision to allocate a DESI fiber on the supernova will be partially based on available information. The available
information for the supernova are be coordinates, current magnitudes, and possibly a metric of transient type.
Information on the purported host are coordinates, magnitudes and possibly a photometric redshift.  
The probability that the supernova gets a DESI observation is $\epsilon^{ST}$, and the probability that there is a  spectrum of the host
after the supernova has faded is $\epsilon^{SH}$.

On a particular date of observation,
DESI has some efficiency of determining the supernova type  $\epsilon^T$ and/or host redshift $\epsilon^z$. 
Generally the spectroscopic redshift uncertainty is sufficiently small to render it irrelevant
for cosmology analysis, and so a successful typing implies a successful redshift.

The probability that the supernova enters a cosmological analysis of spectroscopically confirmed supernova is then
\begin{equation}
\epsilon = \epsilon^D \epsilon^{ST} \epsilon^T,
\end{equation}
and the supernova will have an distance precision of $\sigma_\mu$.

Given a population of supernovae in the universe, we can realize a subset of supernovae that are used in the cosmological analysis.
Figure of merits can be calculated.

\section{Ingredients}




Often, the supernova discovery conditions are sufficiently demanding so that if discovered, then
$\sigma_\mu$ is
dominated by the $\sim 0.12$ mag intrinsic dispersion, rather than measurement uncertainties.


\section{Elements Needed to Determine Science Impact}
Determination of the science yield of a  joint DESI--imaging program can be broken up into conceptually distinct components.
\begin{itemize}
\item Anticipated transient/SN~Ia yield of imaging surveys.

The efficiency of supernova discovery.  The ``average''  supernova at redshift $z$ has light curves $m_X(p;z)$ in filters $X$.  An average supernova
 that explodes at time $t_0$ has  light curves
$m_X(t-t_0; z)$. To first order, the probability of discovering a supernova
in observation $i$ in filter $X_i$ at time $t_i$ is a function of the supernova magnitude  $\epsilon_i(m_{X_i}(t_i-t_0;z))$.  The probability of supernova
discovery is
\begin{equation}
\epsilon(z, t_0) = 1- \sum_{i=1}^{N} \left(1-\epsilon_i(m_{X_i}(t_i-t_0;z))\right),
\end{equation}
where the sum is over all $N$ observations.

The number of supernova discoveries.
The supernova rate is expressed as
\begin{equation}
r(z) = \frac{dN}{d\Omega dz dt}.
\end{equation}
The expected number of supernova discoveries is then
\begin{equation}
 \frac{dN^{D}}{d\Omega dz} = \int_{-\infty}^{\infty} dt_0\, r(z) \epsilon(z, t_0).
\end{equation}




\item Effectiveness of DESI to classify a transient and its redshift, either from the transient or host galaxy.
\item Cosmology Figure of Merit for a set of SNe~Ia with classification, redshift, and imaging-survey photometry.
\item Determination of feasible DESI observing strategy and imaging surveys that optimize Cosmology Figure of Merit 
\end{itemize}

\end{document} 
