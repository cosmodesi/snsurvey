\documentclass[onecolumn]{aastex61}   	% use "amsart" instead of "article" for AMSLaTeX format
\usepackage{geometry}                		% See geometry.pdf to learn the layout options. There are lots.
\geometry{letterpaper}                   		% ... or a4paper or a5paper or ... 
\usepackage{graphicx}				% Use pdf, png, jpg, or eps§ with pdflatex; use eps in DVI mode
\usepackage{amsmath}
\usepackage{amssymb}
\usepackage{natbib}
\usepackage{lineno}
\usepackage{color}
\usepackage{hyperref}
\linenumbers

\begin{document}

\title{DESI Observations of Type Ia Supernovae Discovered by Imaging Surveys}

\begin{abstract}
The DESI survey will be operational concurrently with imaging surveys such as ZTF and LSST.  Due to both higher survey solid-angle coverage and cadence, these imaging surveys are more
efficient than DESI in discovering transients.  Imaging surveys, however,
do not provide accurate classification nor redshifts: information that a spectroscopic instrument such as DESI can provide. The DESI Time Domain
Working Group is therefore determining the potential science yield of a joint DESI--imaging Type~Ia supernova (SN~Ia) program.
\end{abstract}

\section{Fiducial Imaging Survey Plus DESI Program}
Let us consider what happens to one supernova. The supernova is characterized by its intrinsic properties (for example, light-curve
shape, color), its redshift, and is associated with a host galaxy.  It explodes on some observer-frame date $t_0$, and has
a nominal incident flux of $m_X(t)$ through filter $X$ that gets to the observatory. 

An imaging survey takes $N$ observations covering the position of the supernova, at times $t_i$ and filters $X_i$, where $i$ is the observation index.
Transients are searched for in these images, in which the supernova has an opportunity of being detected.
To first order, the probability of detecting a supernova depends on signal-to-noise, such that
in observation $i$ the probability depends on supernova magnitude  $\epsilon_i(m_{X_i}(t_i))$.  The probability of supernova
discovery based on the criterion of a single detection is then
\begin{equation}
\epsilon^D = 1- \sum_{i=1}^{N} \left(1-\epsilon_i(m_{X_i}(t_i))\right),
\end{equation}
where the sum is over all $N$ observations.  If  multiple detections are required for discovery, the efficiency is the sum of all
combinations of detections that allow for discovery.
If the supernova is indeed discovered, the date of discovery $t_D$ is stochastic.  
 
The imaging survey produces light curves of the supernova, from which (together with redshift) its distance can be determined with precision $\sigma_\mu$.
A poor distance determination occurs if the imaging survey does not observe the rise or fall of the supernova.  Therefore,
the supernova is of interest only if its date of explosion $t_0$ occurs from shortly before sky monitoring, up to $\sim 2$ supernova-frame months
before the end of the monitoring. 
 
If the supernova is discovered, there is an opportunity for its targeted DESI observation, and/or the targeted
observation of its host galaxy when the transient light has faded away.   The available
information to inform a DESI fiber allocation for the supernova are its coordinates, current magnitudes, and possibly a metric of transient type.
Information on the purported host are coordinates, magnitudes, and possibly a (photometric) redshift.  
The probability that the supernova gets a DESI observation is $\epsilon^{ST}$, and the probability that the host is obsereved
after the supernova has faded is $\epsilon^{SH}$.

On a particular date of observation,
DESI has some efficiency of determining the supernova type  $\epsilon^T$ and/or host redshift $\epsilon^z$. 
Generally the spectroscopic redshift uncertainty is sufficiently small to render it irrelevant
for cosmology analysis, and so a successful typing implies a successful redshift.

The probability that the supernova enters a cosmological analysis of spectroscopically confirmed supernovae is then
\begin{equation}
\epsilon = \epsilon^D \epsilon^{ST} \epsilon^T,
\end{equation}
and the supernova will have a distance precision of $\sigma_\mu$.

Given a population of supernovae in the universe, we can realize a subset of supernovae that are used in the cosmological analysis.
Science-impact figure of merits can be calculated from subsets.

\section{Ingredients}
This Section picks out the ingredients necessary for the analysis, and fiducial choices for those ingredients.

Supernova coordinates:  Using the rate from Rodney et al (2014),  we realize the $t_0$, $z$, RA and Dec.\ of a supernova.

SN~Ia properties: The supernova is modeled as a SALT2 object.  Given input $t_0$, $z$, we realize a SALT2 supernova,
which can be queried for its observed magnitudes $m_X(t)$.

Survey: The survey is characterized by the set of observations that cover the supernova.  Each observation as a date, filter, and detection efficiency.
Surveys are efficiently approximately characterized by their limiting magnitudes and cadence:
the LSST survey is described by Table 1 of  \url{https://arxiv.org/pdf/0805.2366.pdf} and Equation 1;
depths of the ZTF survey are described in \url{http://www.ptf.caltech.edu/page/ztf_technical} though the cadence remains ambigious.

Detection and Discovery: A fiducial suggestion for detection and discovery efficiency is a single observation of $S/N \ge 7$:
 $\epsilon_i(m_{X_i} \le m_{X, lim}) = 1$,   $\epsilon_i(m_{X_i} > m_{X, lim})=0$, where $m_{X,lim}$ is the $7\sigma$ single-visit limiting
magnitude.  
We realize whether the supernova is discovered and if so on what date $t_D$.

As an interesting intermediate product, we can determine using an ensemble of supernovae the discovery efficiency as a function of redshift $\epsilon^D(z)$, and
the distribution of the phase of discovery $(t_D-t_0)(z)$.

DESI spectroscopic survey: I think (?) that the DESI will cover its footprint
twice in a season, meaning each year there are two possible dates of observation.
For the moment, we can assume
the supernovae gets a fiber if it is located in an area covered by DESI, $\epsilon^{S}=1$; optimization of the DESI follow-up strategy is deferred
for now. 
The efficiency of DESI to get a supernova type and/or redshift, $\epsilon^{ST}$ and  $\epsilon^{SH}$, are to be
determined.  The  $\epsilon^{ST}$ is presumably dependent on the transient properties (i.e.\ magnitude) at the
date of observation.

An interesting product will be $dN/dz$ for the number of supernovae that are spectroscopically typed by DESI, and the
number of supernovae that are not typed but for which there is host redshift.  There are inputs for cosmology figure of merit calculations.

%
%Often, the supernova discovery conditions are sufficiently demanding so that if discovered, then
%$\sigma_\mu$ is
%dominated by the $\sim 0.12$ mag intrinsic dispersion, rather than measurement uncertainties.
%
%
%\section{Elements Needed to Determine Science Impact}
%Determination of the science yield of a  joint DESI--imaging program can be broken up into conceptually distinct components.
%\begin{itemize}
%\item Anticipated transient/SN~Ia yield of imaging surveys.
%
%The efficiency of supernova discovery.  The ``average''  supernova at redshift $z$ has light curves $m_X(p;z)$ in filters $X$.  An average supernova
% that explodes at time $t_0$ has  light curves
%$m_X(t-t_0; z)$. To first order, the probability of discovering a supernova
%in observation $i$ in filter $X_i$ at time $t_i$ is a function of the supernova magnitude  $\epsilon_i(m_{X_i}(t_i-t_0;z))$.  The probability of supernova
%discovery is
%\begin{equation}
%\epsilon(z, t_0) = 1- \sum_{i=1}^{N} \left(1-\epsilon_i(m_{X_i}(t_i-t_0;z))\right),
%\end{equation}
%where the sum is over all $N$ observations.
%
%The number of supernova discoveries.
%The supernova rate is expressed as
%\begin{equation}
%r(z) = \frac{dN}{d\Omega dz dt}.
%\end{equation}
%The expected number of supernova discoveries is then
%\begin{equation}
% \frac{dN^{D}}{d\Omega dz} = \int_{-\infty}^{\infty} dt_0\, r(z) \epsilon(z, t_0).
%\end{equation}
%
%
%
%
%\item Effectiveness of DESI to classify a transient and its redshift, either from the transient or host galaxy.
%\item Cosmology Figure of Merit for a set of SNe~Ia with classification, redshift, and imaging-survey photometry.
%\item Determination of feasible DESI observing strategy and imaging surveys that optimize Cosmology Figure of Merit 
%\end{itemize}

\end{document} 
